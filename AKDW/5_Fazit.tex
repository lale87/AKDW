%---------------------------------- 
\section{Fazit}
%---------------------------------- 
% Verbreitet
Cross-Site-Scripting ist eine weit verbreitete Angriffsform im heutigen
Internet.
% Ziel: Anwender
W�hrend ein Gro�teil der Angriffsszenarien den Server in Visier hat, richtet
sich XSS gegen den Anwender einer Webanwendung.
% Drei Arten
Funktional lassen sich XSS-Angriffe in drei verschiedene Arten einordnen:
persistent, reflektiert und DOM-Basiert.
% Gro�es Gef�hrdungspotential
Besonders Webanwendungen mit sensiblen Daten (Online-Banking oder soziale
Netzwerke) sind anf�llig f�r XSS-Angriffe, da hier ein ein gro�er Schaden
entstehen kann.
% Schutzma�nahmen Client- und Serverseitig
Um diesem entgegen zu wirken m�ssen sowohl server- als auch clientseitig
Ma�nahmen ergriffen werden, welche die Auslieferung bzw. Ausf�hrung von
Schadcode verhindern.
% CSP als L�sung
Als besonders effektiv erweist sich der 2012 standardisierte
Content-Security-Policy-Header (CSP), welcher festlegt, welche Skripte
ausgef�hrt werden d�rfen.
