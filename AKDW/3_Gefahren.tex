%---------------------------------- 
\section{Gefahren}
%---------------------------------- 
% Context of Trust
Was XSS-Angriffe gef�hrlich macht, ist die M�glichkeit sch�dliche Skripte im
sogenannten \enquote{Context of Trust}, der vertrauensw�rdigen Zone einer
Webanwendung, zu starten \cite[1771]{GANA07}.
% Same Origin Policy
So kann ein Angreifer unter anderem auf die Daten der Webanwendung zugreifen,
welche eigentlich durch die \enquote{Same-Origin-Policy} vor Dritten gesch�tzt
sind \cite[276]{Kir11} oder mit h�heren Berechtigungen Skripte ausf�hren.\\

%---------------------------------- 
\subsection{Datendiebstahl}
%----------------------------------
% Daten der Webanwendung
%  Vertrauliche Informationen
Durch den Zugriff auf die Daten der Webanwendung hat der Angreifer je nach
Anwendung Zugang zu vertraulichen Daten. Hierzu k�nnen pers�nliche Daten oder
Firmengeheimnisse geh�ren (z.B. beim Angriff �ber einen Webmailer)
\cite[1773]{GANA07}.
%  Session- & Identit�tsdiebstahl
Zu den sensiblen Daten von Webanwendungen geh�ren unter anderem die
Sitzungsinformationen \cite[276]{Kir11}. Mithilfe dieser kann ein Angreifer eine
Sitzung und somit im weiteren Schritt die Identit�t eines Opfers �bernehmen
\cite[269]{SS06}.
% Tracken
Des Weiteren k�nnen durch ein Skript s�mtliche Eingaben im Browser mitgelesen
werden.

%---------------------------------- 
\subsection{Manipulation von Darstellung oder Information}
%----------------------------------
% Falsche Informationen
Eine weitere Gefahr stellt die Manipulation von Darstellung oder Information auf
Webseiten dar. Inhalte lassen sich ver�ndern oder ersetzen, sodass im Browser
des Anwenders falsche Informationen angezeigt werden. Diese Art des Angriffs
schadet neben dem Anwender auch dem Webseitenbetreiber, da ausgelieferte
Informationen diesem zugeordnet werden (z.B. Imagesch�den f�r Unternehmen)
\cite{Dam06}.
% Werbung
Eine verbreitete Form dieses Angriffs stellt das Platzieren von zus�tzlicher
Werbung auf einer Webseite dar.\\
% Benutzerdaten und Passw�rter
%  "Bitte erneut authentifizieren"
Ebenso verbreitet und mit hohem Gefahrenpotential ist die Einblendung neuer
Dialoge, beispielsweise einem zur erneuten Authentifizierung. Die eingegebenen
Daten kann der Angreifer abgreifen, w�hrend der Anwender in den meisten F�llen
keinen Verdacht sch�pft.

%---------------------------------- 
\subsection{Drive-By-Angriffe}
%----------------------------------
% Guter Einsteigspunkt
%  Durch Webseitenaufruf Code auf lokalem Rechner ausf�hrbar
%  Dive-By-Download
In vielen F�llen agiert ein XSS-Angriff nur als T�r�ffner eines komplexeren
Angriffs. Durch die M�glichkeit allein durch den Aufruf einer Webseite Schadcode
lokal auszuf�hren eignet sich XSS optimal f�r Drive-By-Angriffe.
% Sicherheitsl�cken
Oft in Verbindung mit der Ausnutzung von Sicherheitsl�cken des Browsers k�nnen
so Viren oder Trojaner auf den Rechner des Anwenders geschleust und weiterer
Schaden verursacht werden.

%---------------------------------- 
\subsection{Praktische Relevanz}
%----------------------------------
% Beispiele / Relevanz
XSS-Angriffe sind in der Praxis weit verbreitet. Besonders Online-Banken
(Datendiebstahl) und soziale Netzwerke (Identit�ts�bernahme) geraten oft ins
Fadenkreuz der Angreifer, da hier besonders sensible Daten abgegriffen werden
k�nnen. In der Vergangenheit waren unter anderem PayPal \cite{Kug13}, Facebook
\cite{Gol13}, Google \cite{Shi05}, die Webpr�senz der Bundesregierung
\cite{Dam06} und verschiedene Online-Banken \cite{Sto10} betroffen.
