% TODO: Gefahren
%---------------------------------- 
\section{Gefahren}
%---------------------------------- 
Die Gefahren, die durch XSS entstehen, sind vielf�ltig. Ist das Einschleusen von Schadcode 
erfolgreich, steht in der Regel das gesamte Funktionsspektrum der jeweiligen
Skript-Sprache zur Verf�gung, das f�r einen Angriff genutzt werden kann.

%---------------------------------- 
\subsection{Programme nachladen}
%----------------------------------
%TODO: Drive-By-Download
Ddadurch, dass der Schadcode clientseitig ausgef�hrt wird, ist es m�glich durch
diesen Programme von externen Servern nachzuladen, die weiteren Schadcode
enthalten. Es ist also nicht notwendig, das der Nutzer aktiv einen Download
durchf�hrt (Drive-By-Download). Nachgeladen werden k�nnen hier�ber z.B.Trojaner,
die noch weiteren Schaden verursachen k�nnen.
%---------------------------------- 
\subsection{Informationsdiebstahl}
%---------------------------------- 
%TODO: Informationsdiebastahl
Die M�glichkeit Informationen zu stehlen sind vielf�ltig. Neben der �ffnung
eines Dialogs, der den Nutzer auffordert, sich durch vertrauliche Informationen
erneut zu identifizieren, kann auch direkt im Kontext der Anwendung auf
Session-Informationen zugegriffen werden, was wiederum die M�glichkeit eines
Identit�tsdiebstahls erm�glicht. Eine weiter M�glichkeit, Informationen zu
steheln ist das Tracken s�mtlich Nutzereingaben und das Senden dieser an einen
externen Server. Die so angegriffenen Informationen k�nnen vertraulichen Inhalt
wie Firmengeheimnisse und auch Login-Daten beinhalten 
%---------------------------------- 
\subsection{Verbreitung von Fehlinformationen}
%----------------------------------
%TODO: Fehlinformationen
Mit Hilfe von Skripten ist es m�glich, eigene Texte auf einer fremden Webseite
einzubinden. Das Angriffsziel hierbei ist weiterhin der Client, indem dieser
falsche Informationen erh�lt, allerdings kann es auch dazu genutzt werden, dem
Betreiber des Webservers zu schaden, da die Fehlinformation durch seinen
Auftritt verbreitet wird und als Unternehmensaussage interpretiert werden kann.


% Beispiele
%
% PayPal
%  http://www.heise.de/security/meldung/PayPal-wieder-durch-Cross-Site-Scripting-angreifbar-1869515.html
% Facebook
%  http://www.heise.de/security/meldung/Facebook-schliesst-Cross-Site-Scripting-Luecken-1845343.html
% Bundesregierung
%  http://www.spiegel.de/netzwelt/tech/angeblicher-merkel-ruecktritt-luecke-in-website-der-bundesregierung-a-435128.html
% Online-Banking
%  http://www.spiegel.de/netzwelt/web/online-banking-16-jaehriger-findet-17-banken-sicherheitsluecken-a-722049.html
% Ebay
%  http://www.spiegel.de/netzwelt/web/ebay-passwortwechsel-nach-hackerangriff-weitere-sicherheitsprobleme-a-971661.html
% Google
%  http://shiflett.org/blog/2005/dec/googles-xss-vulnerability
