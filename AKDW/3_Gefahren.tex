% TODO: Gefahren
%---------------------------------- 
\section{Gefahren}
%---------------------------------- 
Die Gefahren, die durch XSS entstehen, sind vielf�ltig. Ist das Einschleusen von Schadcode 
erfolgreich, steht i.d.R. das gesamte Funktionsspektrum der jeweiligen Script-Sprache zur Veer�gung, 
das f�r einen Angriff genutzt werden kann. Nachfolgend aufgelistet finden sich
lediglich einige wenige Angriffsszenarien, die sich mit Hilfe von XSS umsetzen lassen.

%---------------------------------- 
\subsection{Cookie Harvesting}
%---------------------------------- 
Dadurch, dass das Script Clientseitig ausgef�hrt wird, ist es m�glich auf die lokalen 
Dateien zuzugreifen und diese zu kopieren und �ber das Netzwerk (ggf. Internet) zu �bertragen. 
Auf diese Art und Weise ist es m�glich, durch ein Skript Cookies zu stehlen. Die Cookies werden 
hierbei an den Angreifer �bertragen und dieser erh�lt somit im schlechtesten Fall alle aktiven 
Logins des Opfers.
%---------------------------------- 
\subsection{Informationsdiebstahl}
%---------------------------------- 
Die Erzeugung eines Eingabefensters kann daf�r genutzt werden, (vertrauliche) Information vom 
Opfer zu bekommen. Ein Angriffsszenario hierbei ist das Abfragen der Login-Daten mit dem Hinweis, 
dass die Session abgelaufen ist und man sich erneut anmelden m�sse.  
%---------------------------------- 
\subsection{Key-Logger}
%----------------------------------
Es ist m�glich mit Hilfe eines Scripts die Eingabe der Tastatur zu loggen und an den Angreifer zu �bertragen. 
Auf diese Weise k�nnen Zugangsdaten abgegriffen werden.
%---------------------------------- 
\subsection{Weiteres Beispiel}
%----------------------------------
Hier k�nnte noch ein Beispiel kommen\ldots

% Beispiele
%
% PayPal
%  http://www.heise.de/security/meldung/PayPal-wieder-durch-Cross-Site-Scripting-angreifbar-1869515.html
% Facebook
%  http://www.heise.de/security/meldung/Facebook-schliesst-Cross-Site-Scripting-Luecken-1845343.html
% Bundesregierung
%  http://www.spiegel.de/netzwelt/tech/angeblicher-merkel-ruecktritt-luecke-in-website-der-bundesregierung-a-435128.html
% Online-Banking
%  http://www.spiegel.de/netzwelt/web/online-banking-16-jaehriger-findet-17-banken-sicherheitsluecken-a-722049.html
% Ebay
%  http://www.spiegel.de/netzwelt/web/ebay-passwortwechsel-nach-hackerangriff-weitere-sicherheitsprobleme-a-971661.html
% Google
%  http://shiflett.org/blog/2005/dec/googles-xss-vulnerability
