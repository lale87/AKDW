%---------------------------------- 
\subsection{Schutzma�nahmen -- Server}
%---------------------------------- 
% Nutzereingaben pr�fen
Im Allgemeinen treten XSS-L�cken bei nicht ausreichend validierten
Benutzereingaben auf, die zur Aufbereitung dynamischer Webseiten genutzt werden.
Die �berpr�fung aller Benutzereingaben direkt vor der Generierung einer Seite
spielt also eine wichtige Rolle. Aufgrund ihrer Funktionsweise lassen sich keine
Eingaben durch DOM-basiertes XSS serverseitig �berpr�fen.
% Filter-Proxy
Eine zentrale M�glichkeit stellen Filter-Proxies dar. Diese scannen ein- und
ausgehende Datenstr�me auf entsprechenden Schadcode anhand von Heuristiken
\cite[1776ff.]{GANA07} \cite{Fox12}.
% Problem & DOM-XSS
Grundproblem dieser Ma�nahmen ist, dass der Angreifer nur eine L�cke im
Schutzsystem finden muss, w�hrend der Betreiber alle M�glichkeiten zu
ber�cksichtigen hat.\\
% Content-Security-Policy
Eine relative junge Entwicklung ist das setzen des
Content-Security-Policy-Headers (CSP). Dieser wird mittlerweile von allen
g�ngigen Browsern unterst�tzt und kann XSS-Angriffe effektiv verhindern. In
diesem Header lassen sich vertrauensw�rdige Ort von Inhalten festlegen. Indem
eigene Skripte vollst�ndig in externe Dateien ausgelagert werden, lassen sich
Inline-Skripte komplett verbieten und nachgeladene Skripte auf Orte der eigenen
Domain beschr�nken. Umsetzen muss die Regeln der Browser des Anwenders
\cite{SB12} \cite[1780ff.]{GANA07}.
