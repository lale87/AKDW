%---------------------------------- 
\subsection{Schutzma�nahmen -- Server}
%---------------------------------- 
% Nutzereingaben pr�fen
Im Allgemeinen treten XSS-L�cken nur bei nicht ausreichend validierten
Benutzereingaben auf, die zur Aufbereitung dynamischer Webseiten genutzt werden.
Die �berpr�fung aller Benutzereingaben direkt vor der Generierung einer Seite
spielt also eine wichtige Rolle. Dieser Vorgang ist jedoch nicht trivial. Durch
viele verschiedene Einbindungsm�glichkeiten f�r Skripte ..\\
% Filter-Proxy
Eine weitere M�glichkeit stellen zentrale Filter-Proxies dar. Diese scannen
ein- und ausgehende Datenstr�me auf entsprechenden Schadcode anhand von
Heuristiken.
% Policies
Eine relative junge Entwicklung ist das Vereinbaren von Policies mit dem
Browser. Zu jeder Webseite wird �ber eine gesicherte Verbindung eine Art Vertrag
ausgetauscht. Dieser beschreibt, welche Aktionen f�r bestimmte Skriptquellen
erlaubt sind. So l�sst sich das Nachladen von externen Skripten unterbinden und
nicht ben�tigte Funktionen deaktivieren. Dieses Verfahren muss jedoch vom
Browser auf dem Client umgesetzt werden.\\
% TODO: Umsetzung? Verbreitung?
% Problem & DOM-XSS
Trotz aller Schutzma�nahmen gibt es ein Grundproblem: Der Angreifer braucht nur
eine L�cke im Schutzsystem zu finden, w�hrend der Webseitenbetreiber aller
M�glichkeiten ber�cksichtigen muss.\\
% Ganzheitlicher Ansatz / sichere Architektur
Allgemein gilt es deswegen einen ganzheitlichen, auf Sicherheit ausgelegten
Ansatz zu fahren und die Anwendung gezielt gegen Angriffe zu designen. Hierbei
hilft der Einsatz von Frameworks sehr, da hier die meisten bekannten XSS-L�cken
ber�cksichtigt werden.
% TODO: Apache?
