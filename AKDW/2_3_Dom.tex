%---------------------------------- 
\subsection{Document Object Model (DOM)-Basiert (lokal)}
%----------------------------------
DOM-Basiertes XSS nutzt URL-Parameter, um Schadcode auf einem fremden Client zur
Ausf�hrung zu bringen. Die Besonderheit in dieser Methode liegt darin, dass der
im URL-Parameter eingebettete Schadcode nicht vom Server ausgelesen und in die
Webseite eingebettet wird, sondern erst durch den Client \cite{fSidI13}. 
Das Einbetten des Schadcodes in die Webseite erfolgt dadurch, dass
Schwachstellen in dem unver�nderten Skript der Webseite ausgenutzt werden. Die
mit der URL gelieferten Parameter werden von dem Skript der Webseite ausgelesen
und f�r die Darstellung verwendet. Wird nun eine Skript-Routine als
Parameter �bergeben und der Parameter durch das Skript der Webseite
ungefiltert zur Anzeige gebracht, so wird der Schadcode auf dem Client
ausgef�hrt \cite{Fox12}.\\
\nolinkurl{http://www.trustedsite.tld/welcome}\bfurl{#name=<script>alert('HACKED')</script>}
