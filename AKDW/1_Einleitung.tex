\section{Einleitung}
Cross-Site-Scripting, auch als XSS bekannt, beruht darauf, dass eine dynamisch
generierte Webseite Benutzereingaben nicht korrekt validiert und darstellt.
Der Angreifer versucht auf diese Weise ein eigenes Skript in die Webseite
einzubinden, welches beim Anwender ausgef�hrt wird. Da dieses nun im Kontext
einer vermeintlich sicheren Webseite ausgef�hrt wird, kann der Angreifer
meistens auf sensible Daten der Webanwendung zugreifen \cite[1771]{GANA07}.\\
% Definition
Das Bundesamt f�r Sicherheit in der Informationstechnik definiert XSS
folgenderma�en:\\
\textit{\enquote{Cross-Site Scripting-Angriffe
(XSS-Angriffe) richten sich gegen die Benutzer einer Webanwendung und deren
Clients. Hierbei versucht ein Angreifer indirekt Schadcode (in der Regel
Browser-seitig ausf�hrbare Skripte, wie z. B. JavaScript) an den Client des
Benutzers der Webanwendung zu senden.}} \cite{fSidI13}\\
% Ziel und Abgrenzung
Ziel eines solchen Angriffs, in Abgrenzung zur SQL-Injection, ist nicht die
Webanwendung selbst, sondern der Anwender \cite{Fox12}. In diesem Papier wird
als Beispiel JavaScript herangezogen, welches bei XSS-Angriffen am h�ufigsten
zum Einsatz kommt. Zu beachten ist jedoch, dass grunds�tzlich jede vom Browser
ausf�hrbare Markup- oder Skriptsprache (HTML, CSS, Flash, ActiveX, etc.) genutzt
werden kann.
