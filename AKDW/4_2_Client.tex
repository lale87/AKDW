%---------------------------------- 
\subsection{Schutzma�nahmen -- Client}
%---------------------------------- 
%allgemein
Clientseitige Schutzma�nahmen m�ssen ergriffen werden, da nicht zwangsl�ufig
alle Schwachstellen serverseitig behoben werden und dadurch an den Client
ausgeliefert werden k�nnen. 
Schutzma�nahmen k�nnen darauf abzielen, das Einbinden des Skripts oder
auch das Ausf�hren des Skripts zu unterbinden.
%Skripte nicht einbinden 
Das Einbinden von Schadcode l�sst sich clientseitig nur bei refklektiertem und
DOM-Basiertem XSS verhindern, da bei persistentem XSS der Schadcode ohne Beteiligung des Clients zuvor
eingebunden wurde \cite[1772ff]{GANA07}.
Bei den erstgenannten Methoden wird er Angriff durch eine manipulierte Url
durchgef�hrt, sodass diese clientseitig auf ungew�hnliche Skript-Elemente
�berpr�ft werden kann.
%Skripte blocken
Der effektivste Schutzt gegen XSS ist clientseitig das Unterbinden von Skript.
Das grunds�tzliche Blocken aller Skript-Routinen ist hierbei das sicherste
Element im schutz gegen XSS. Da heutzutage allerdings viele Funktionali�ten im
Internet durch Skript-Routinen erst m�glich werden, f�hrt ein grunds�tzliches
Blocken der Inhalte dazu, dass viele Webseiten im Internet f�r die Nutzer nur
noch sehr eingeschr�nkt nutzbar sind \cite{Fox12}. Um Skripte nur selektiv zu
unterbinden, gibt es Programme, die innerhalb des Browser Webseiten auf
auff�llige Skript-Routinen untersuchen und diese blockieren, wobei auch diese
keinen hundertprozentigen Schutz bieten k�nnen, da das Angriffsspektrum zu gro�
ist \cite{Fox12}.

