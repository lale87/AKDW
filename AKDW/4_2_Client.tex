%---------------------------------- 
\subsection{Schutzma�nahmen -- Client}
%---------------------------------- 
Clientseitige Schutzma�nahmen m�ssen ergriffen werden, da nicht zwangsl�ufig
alle Schwachstellen serverseitig behoben werden und dadurch an den Client
ausgeliefert werden k�nnen. 
% Alle Angriffe nutzen diese Schwachstellen aus, so
% werden auch beim DOM-Basierte XSS die Schwachstellen in dem vom
% Server ausgelieferten Script ausgenutzt, um einen Angriff durchzuf�hren.
Schutzma�nahmen k�nnen sowohl darauf abzielen, das Einbinden des Skripts oder
auch das Ausf�hren des Skripts zu unterbinden. Das Einbinden von Schadcode l�sst
sich clientseitig nur bei refklektiertem und DOM-Basiertem XSS verhindern, da
bei persistentem XSS der Schadcode ohne Beteiligung des Clients zuvor
eingebunden wurde \cite[1772ff]{GANA07}.
Bei den erstgenannten Methoden wird er Angriff durch eine manipulierte Url
durchgef�hrt, sodass diese clientseitig auf ungew�hnliche Skript-Elemente
�berpr�ft werden kann.

Der effektivste Schutzt gegen XSS ist clientseitig das Unterbinden von Skript.
Das grunds�tzliche Blocken aller Skript-Routinen ist hierbei das sicherste
Element im schutz gegen XSS. Da heutzutage allerdings viele Funktionali�ten im
Internet durch Skript-Routinen erst m�glich werden, f�hrt ein grunds�tzliches
Blocken der Inhalte dazu, dass viele Webseiten im Internet f�r die Nutzer nur
noch sehr eingeschr�nkt nutzbar sind \cite{Fox12}. Um Skripte nur selektiv zu
unterbinden, gibt es Programme, die innerhalb des Browser Webseiten auf
auff�llige Skript-Routinen untersuchen und diese blockieren \cite{Fox12}. 

%---------------------------
% Die Ma�nahmen, die clientseitig getroffen werden k�nnen, um einen Angiff mittels Cross-Site-Scripting zu verhindern, beschr�nken sich prim�r auf Vorsichtsma�nahmen und die Sicherheitseinstellungen im genutzten Browser.
% Die effektivste methode, um jegiche Art von Cross-Site-Scripting zu verhindern
% ist das Blocken s�mlticher Scripte von Websiten. Die Problematik hierbei ist,
% dass damit auch s�mtliche Scripte geblockt werden, die zur normalen Ausf�hrung
% einer Website notwendig sein k�nnen; die Nutzung des Internets ist somit nur
% noch sehr eingeschr�nkt m�glich. 
% Neben der Blockierung von Script-Routinen muss beim Aktivieren von Links darauf
% geachtet werde, welche Paramameter in der Url enthalten sind. Scripte-Elemente
% innerhalb einer Url sind ein Indiz f�r einen m�glichen Agriff durch
% Cross-Site-Scripting.
% Dar�ber hinaus besteht die M�glichkeit, die die Netzweraktivit�ten des Clients
% zu �berwachen. Cross-Site-Scripting basierte Angriffe f�hren in der Regel zu
% ungew�hnlichen Netzwerkaktivit�ten in Form von Datenaustausch mit unbekannten
% Netzwerkteilnehmern. Eine Identifikation solcher Aktivit�ten ist aufgrund der
% hohen Kommunikationsrate in der Praxis eher unzuverl�ssig.
