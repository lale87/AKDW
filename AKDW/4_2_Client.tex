%---------------------------------- 
\subsection{Schutzma�nahmen -- Client}
%---------------------------------- 
%allgemein
Clientseitige Schutzma�nahmen m�ssen ergriffen werden, da nicht alle
Schwachstellen serverseitig ber�cksichtigt werden k�nnen (DOM-Basiertes XSS).
Schutzma�nahmen k�n\-nen darauf abzielen, das Einbinden oder auch das Ausf�hren
eines Skripts zu unterbinden.
%Skripte nicht einbinden 
Das Einbinden von Schadcode l�sst sich clientseitig nur bei reflektiertem und
DOM-Basiertem XSS verhindern, da bei persistentem XSS der Schadcode ohne
Beteiligung des Clients zuvor eingebunden wurde \cite[1772ff]{GANA07}.
Bei den erstgenannten Methoden wird der Angriff �ber eine manipulierte URL
durchgef�hrt, so dass diese clientseitig auf ungew�hnliche Skript-Elemente
�berpr�ft werden kann.
%Skripte blocken
Der effektivste Schutz gegen XSS ist das clientseitige Unterbinden von Skripten.
Da heutzutage allerdings viele Funktionalit�ten im Internet durch
Skript-Routinen erst m�glich werden, f�hrt ein grunds�tzliches Blocken der
Inhalte dazu, dass viele Webseiten im Internet f�r die Nutzer nur noch sehr
eingeschr�nkt nutzbar sind \cite{Fox12}. Um Skripte nur selektiv zu unterbinden,
gibt es Programme, die innerhalb des Browsers Webseiten auf auff�llige
Skript-Routinen untersuchen und diese blockieren, wobei auch diese keinen
hundertprozentigen Schutz bieten k�nnen, da das Angriffsspektrum zu gro� ist
\cite{Fox12}.
