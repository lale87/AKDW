%---------------------------------- 
\subsection{Clientseitig}
%---------------------------------- 
Die Ma�nahmen, die clientseitig getroffen werden k�nnen, um einen Angiff
mittels Cross-Site-Scripting zu verhindern, beschr�nken sich prim�r auf
Vorsichtsma�nahmen und die Sicherheitseinstellungen im genutzten Browser.
Die effektivste methode, um jegiche Art von Cross-Site-Scripting zu verhindern
ist das Blocken s�mlticher Scripte von Websiten. Die Problematik hierbei ist,
dass damit auch s�mtliche Scripte geblockt werden, die zur normalen Ausf�hrung
einer Website notwendig sein k�nnen; die Nutzung des Internets ist somit nur
noch sehr eingeschr�nkt m�glich. 
Neben der Blockierung von Script-Routinen muss beim Aktivieren von Links darauf
geachtet werde, welche Paramameter in der Url enthalten sind. Scripte-Elemente
innerhalb einer Url sind ein Indiz f�r einen m�glichen Agriff durch
Cross-Site-Scripting.
Dar�ber hinaus besteht die M�glichkeit, die die Netzweraktivit�ten des Clients
zu �berwachen. Cross-Site-Scripting basierte Angriffe f�hren in der Regel zu
ungew�hnlichen Netzwerkaktivit�ten in Form von Datenaustausch mit unbekannten
Netzwerkteilnehmern. Eine Identifikation solcher Aktivit�ten ist aufgrund der
hohen Kommunikationsrate in der Praxis eher unzuverl�ssig.
