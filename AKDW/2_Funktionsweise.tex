%---------------------------------- 
\section{Funktionsweise}
%---------------------------------- 
XSS beruht darauf, dass ein Skript beim Aufruf einer 
Website ausgef�hrt wird, dass nicht urspr�nglich vom 
Websitebetreiber stammt. Du Funktionsweise, wie die 
Ausf�hrung des Skripts hervorgerufen werden kann, ist vielf�ltig.
Das Bundesamt f�r Sicherheit in der Informationstechnik 
klassifiziert diese unterschiedlichen Funktionsweisen in 3 Bereiche.
%---------------------------------- 
\subsection{Persistent}
%---------------------------------- 
Persistentes XSS zielt darauf ab, Schadcode in Form einer Skript-Routine
dauerhaft auf einem Webserver zu speichern.
% , sodass der Schadcode an jeden
% Nutzer ausgeliefert wird, der die Infizierte Seite besucht\cite{Fox12}.
Um den Schadcode auf den Server zu �bertragen werden Schnittstellen benutzt, die
es einem Nutzer erlauben, eigene Eingaben an einen Server zu senden, die in der
sp�teren Darstellung einer Webseite genutzt werden. Der hier eingebene Text,
wie er z.B. in Diskussionforen h�ufig vorkommt, kann neben dem eigentlichen Text
um Skript-Elemente erweitert werden, die dem Text angeh�ngt
werden \cite{fSidI13}. Werden die �bertragenen Eingabedaten nicht gefiltert, so
wird der Schadcode dauerhaft in die Seite eingebettet.
Jeder Besucher der so infizierten Webseite bekommt vom vermeintlich sicheren
Server nun den Schadcode ausgeliefert, der vom Browser des Nutzers interpr�tiert
und ausgef�hrt wird, sofern dies nicht speziell vom Client unterbunden
wird \cite[1772]{GANA07}.

%---------------------------------- 
\subsection{Reflektiert}
%---------------------------------- 
% Allgemeine Beschreibung
Reflektiertes XSS nutzt die URL zur Einbettung sch�dlicher Skripte
\cite[1773]{GANA07}. In den meisten F�llen werden hierbei GET-Parameter genutzt.
Im Gegensatz zu POST-Parametern, welche im Hintergrund �bertragen werden, h�ngen
GET-Parameter direkt an der aufgerufenen URL \cite[56]{SH06}. Durch diese ist es
m�glich direkt auf dynamische Webseiten zu verlinken (z. B. einen bestimmten
Artikel im Webshop). Der Aufruf einer URL mit GET-Parametern ist weit verbreitet
und sieht folgenderma�en aus:\\
\nolinkurl{http://www.webshop.de/show_article.php}\bfurl{?productid=4711}\\
Hierbei wird die dynamische Seite \textit{show\_article.php} mit dem Parameter
\textit{productid=4711} aufgerufen. 
% Vorgehen
Problematisch wird es, wenn manipulierte Parameter oder Teile der URL zur
Aufbereitung der Webseite genutzt werden. Ein Angreifer kann so Einfluss auf die
Aufbereitung einer Webseite beim Anwender nehmen, indem er eine entsprechend
manipulierte URL bereitstellt \cite{Fox12}. So wird beispielsweise oft bei
Fehlermeldungen, wenn eine Seite nicht aufgerufen werden konnte, die URL im Text
angezeigt. Durch die Integration eines \texttt{<script>}-Blocks in die URL
lassen sich externe Skripte einbinden und im aufrufenden Browser ausf�hren. Der
Angreifer muss nur das Opfer dazu bewegen die manipulierte URL zu �ffnen.
% Vom Server "reflektiert"
Da der Server beim Aufruf der manipulierten URL die M�glichkeit hat die
angeh�ngten Parameter auf sch�dliche Werte zu �berpr�fen bevor er etwas an den
Client ausliefert, ist diese Art als reflektiertes XSS bekannt. Im Gegensatz zum
persistenten XSS liefert der Server die schadhaften Skripte nur beim Aufruf der
entsprechenden manipulierten URL aus \cite[276]{Kir11}.

%---------------------------------- 
\subsection{Document Object Model (DOM)-Basiert (lokal)}
%----------------------------------
DOM-Basiertes XSS nutzt Url-Parameter, um Schadcode auf einem fremden Client zur
Ausf�hrung zu bringen. Die Besonderheit in dieser Methode liegt darin, dass der
im Url-parameter eingebettete Schadcode nicht vom Server ausgelesen und in die
Webseite eingebettet wird, sondern erst durch den Client \cite{fSidI13}. 
Das Einbetten des Schadcodes in die Webseite erfolgt dadurch, dass
Schwachstellen in dem unver�nderten Skript der Webseite ausgenutzt werden. Die
mit der Url gelieferten Parameter werden von dem Skript der Webseite ausgelesen
und f�r die Darstellung verwendet. Wird nun eine Skript-Routine als
Parameter �bergeben und der Parameter wird durch das Skript der Webseite
ungefiltert zur Anzeige gebracht, so wird das Skript auf dem Client
ausgef�hrt \cite{Fox12}. \\
\texttt{http://www.trustedsite.tld/welcome\#name=<script>alert('HACKED')</script>}\\

% %---------------------------------- 
% \subsection{Persistent}
% %---------------------------------- 
% Beim persistenten XSS wird das Script durch den Angreifer auf einen Webserver
% �bertragen und dort persistiert. Durch diese dauerhafte Speicherung des Schadcodes 
% wird das Script bei jedem Seitenaufruf direkt von dem Webserver an die Clients 
% ausgeliefert. Ein gutes Beispiel hierf�r ist das �bertragen von Schadcode in Foren. 
% Durch das Einbinden von Scripten in Kommentare, wird dieses auf den Server �bertragen 
% von dort an jeden Leser wieder ausgeliefert. Es Bedarf also keiner aktiven T�tigkeit 
% durch den Anwender mit Ausnahme des Besuchs der vermeintlich sicheren Seite.
% %---------------------------------- 
% \subsection{Reflektiert}
% %---------------------------------- 
% Das reflektierte XSS nutzt die Manipulation von GET-Paramtern, um das Script zur
% Ausf�hrung zu bringen. Der GET-Paramater wird um Script-Elemente erweitert und beim 
% Absenden an den Server auf diesen �bertragen. Das Script bleibt in diesem fall allerdings 
% nicht dauerhaft auf dem Server erhalten. Betroffen ist als nur der Anwender, der diesen 
% manipulierten GET-Parameter mit der Url absendet. Durch das Senden dieses manipulierten 
% Requests kommt es zu einer Auslieferung des ausf�hrbaren Scripts an den Anwender.  
% Um nun einen Anwender dazu zu bringen, diesen manipulierten Request zu senden, versucht 
% der Angreifer i.d.R. einen manipulierten Link zu verteilen, der diesen Parameter enth�lt. 
% %---------------------------------- 
% \subsection{DOM-Basiert (lokal)}
% %----------------------------------
% Wie auch beim reflektierten XSS wird beim DOM-Basierten XSS ein Parameter in der
%  Url genutzt, um ein Script beim Anwender zur Ausf�hrung zu bringen. Im Gegensatz zum 
%  reflektierten XSS wird das Script allerdings nicht an den Server gesendet, sondern direkt 
%  beim Anwender ausgef�hrt, es findet also nur lokal statt. Der Angriff erfolgt i.d.R. 
%  ebenfalls �ber einen manipulierten Link.
