%---------------------------------- 
\section{Funktionsweise}
%---------------------------------- 
XSS beruht darauf, dass ein Skript beim Aufruf einer 
Website ausgef�hrt wird, dass nicht urspr�nglich vom 
Websitebetreiber stammt. Du Funktionsweise, wie die 
Ausf�hrung des Skripts hervorgerufen werden kann, ist vielf�ltig.
Das Bundesamt f�r Sicherheit in der Informationstechnik 
klassifiziert diese unterschiedlichen Funktionsweisen in 3 Bereiche.
%---------------------------------- 
\subsection{Persistent}
%---------------------------------- 
Beim persistenten XSS wird das Script durch den Angreifer auf einen Webserver
�bertragen und dort persistiert. Durch diese dauerhafte Speicherung des Schadcodes 
wird das Script bei jedem Seitenaufruf direkt von dem Webserver an die Clients 
ausgeliefert. Ein gutes Beispiel hierf�r ist das �bertragen von Schadcode in Foren. 
Durch das Einbinden von Scripten in Kommentare, wird dieses auf den Server �bertragen 
von dort an jeden Leser wieder ausgeliefert. Es Bedarf also keiner aktiven T�tigkeit 
durch den Anwender mit Ausnahme des Besuchs der vermeintlich sicheren Seite.
%---------------------------------- 
\subsection{Reflektiert}
%---------------------------------- 
Das reflektierte XSS nutzt die Manipulation von GET-Paramtern, um das Script zur
Ausf�hrung zu bringen. Der GET-Paramater wird um Script-Elemente erweitert und beim 
Absenden an den Server auf diesen �bertragen. Das Script bleibt in diesem fall allerdings 
nicht dauerhaft auf dem Server erhalten. Betroffen ist als nur der Anwender, der diesen 
manipulierten GET-Parameter mit der Url absendet. Durch das Senden dieses manipulierten 
Requests kommt es zu einer Auslieferung des ausf�hrbaren Scripts an den Anwender.  
Um nun einen Anwender dazu zu bringen, diesen manipulierten Request zu senden, versucht 
der Angreifer i.d.R. einen manipulierten Link zu verteilen, der diesen Parameter enth�lt. 
%---------------------------------- 
\subsection{DOM-Basiert (lokal)}
%----------------------------------
Wie auch beim reflektierten XSS wird beim DOM-Basierten XSS ein Parameter in der
 Url genutzt, um ein Script beim Anwender zur Ausf�hrung zu bringen. Im Gegensatz zum 
 reflektierten XSS wird das Script allerdings nicht an den Server gesendet, sondern direkt 
 beim Anwender ausgef�hrt, es findet also nur lokal statt. Der Angriff erfolgt i.d.R. 
 ebenfalls �ber einen manipulierten Link.
