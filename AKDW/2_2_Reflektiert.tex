%---------------------------------- 
\subsection{Reflektiert}
%---------------------------------- 
% Allgemeine Beschreibung
Reflektiertes XSS nutzt an die URL angeh�ngte Parameter zur Einbettung
sch�dlicher Skripte. Im Gegensatz zu POST-Parametern, welche im Hintergrund
�bertragen werden, h�ngen GET-Parameter direkt an der aufgerufenen URL. Durch
diese ist es m�glich direkt auf dynamische Webseiten zu verlinken (z. B.
einen bestimmten Artikel im Webshop). Der Aufruf einer URL mit GET-Parametern
ist weit verbreitet und sieht folgenderma�en aus:\\
\texttt{http://www.webshop.de/show\_article.php\textbf{?productid=4711}}\\
Hierbei wird die dynamische Seite \textit{show\_article.php} mit dem Parameter
\textit{productid=4711} aufgerufen. 
% Vorgehen
Problematisch wird es, wenn manipulierte Parameter zur Aufbereitung einer
Webseite genutzt werden. Ein Angreifer kann so Einfluss auf die Aufbereitung
einer Webseite beim Anwender nehmen, indem er eine entsprechend manipulierte URL
bereitstellt. So wird beispielsweise oft bei Fehlermeldungen, wenn eine Seite
nicht aufgerufen werden konnte, die URL im Text angezeigt. Durch integrieren
eines \texttt{<script>}-Blocks in der URL lassen sich externe Scripte einbinden
und beim aufrufenden Client ausf�hren. Der Angreifer muss nun nur noch das Opfer
dazu bewegen diese manipulierte URL zu �ffnen.
% Vom Server "reflektiert"
Da der Server beim Aufruf der manipulierten URL die M�glichkeit hat die
angeh�ngten Parameter auf sch�dliche Werte zu �berpr�fen und selbst an den
Client ausliefert, ist diese Art als reflektiertes XSS bekannt. Im Gegensatz zum
persistenten XSS liefert der Server die schadhaften Scripte nur beim Aufruf der
entsprechenden manipulierten URL aus.
