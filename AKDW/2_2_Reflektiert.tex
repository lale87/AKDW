%---------------------------------- 
\subsection{Reflektiert}
%---------------------------------- 
% Allgemeine Beschreibung
Reflektiertes XSS nutzt die URL zur Einbettung sch�dlicher Skripte
\cite[1773]{GANA07}. In den meisten F�llen werden hierbei GET-Parameter genutzt.
Im Gegensatz zu POST-Parametern, welche im Hintergrund �bertragen werden, h�ngen
GET-Parameter direkt an der aufgerufenen URL \cite[56]{SH06}. Durch diese ist es
m�glich direkt auf dynamische Webseiten zu verlinken (z. B. einen bestimmten
Artikel im Webshop). Der Aufruf einer URL mit GET-Parametern ist weit verbreitet
und sieht folgenderma�en aus:\\
\nolinkurl{http://www.webshop.de/show_article.php}\bfurl{?productid=4711}\\
Hierbei wird die dynamische Seite \textit{show\_article.php} mit dem Parameter
\textit{productid=4711} aufgerufen. 
% Vorgehen
Problematisch wird es, wenn manipulierte Parameter oder Teile der URL zur
Aufbereitung der Webseite genutzt werden. Ein Angreifer kann so Einfluss auf die
Aufbereitung einer Webseite beim Anwender nehmen, indem er eine entsprechend
manipulierte URL bereitstellt \cite{Fox12}. So wird beispielsweise oft bei
Fehlermeldungen, wenn eine Seite nicht aufgerufen werden konnte, die URL im Text
angezeigt. Durch die Integration eines \texttt{<script>}-Blocks in die URL
lassen sich externe Skripte einbinden und im aufrufenden Browser ausf�hren. Der
Angreifer muss nur das Opfer dazu bewegen die manipulierte URL zu �ffnen.
% Vom Server "reflektiert"
Da der Server beim Aufruf einer manipulierten URL die M�glichkeit hat die
angeh�ngten Parameter auf sch�dliche Werte zu �berpr�fen bevor er etwas an den
Client ausliefert, ist diese Art als reflektiertes XSS bekannt. Im Gegensatz zum
persistenten XSS liefert der Server die schadhaften Skripte nur beim Aufruf der
entsprechend manipulierten URL aus \cite[276]{Kir11}.
